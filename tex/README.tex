\newcommand\Python{\emph{Python}}
\newcommand\Git{\emph{Git}}
\newcommand\Mercurial{\emph{Mercurial}}
\newcommand\Subversion{\emph{Subversion}}
\newcommand\git{\lstinline{git}}
\newcommand\hg{\lstinline{hg}}
\newcommand\svn{\lstinline{svn}}
\newcommand\Linux{\emph{Linux}}
\newcommand\Windows{\emph{Windows}}
\newcommand\Mac{\emph{Mac}}
\newcommand\onsubpy{\lstinline{onsub.py}}

\input{fullheader.tex}

\title{onsub.py}

\begin{document}

\section*{Abstract}
The \onsubpy\ command iterates over subfolders to execute arbitrary commands as dictated by command line arguments and a \Python\ configuration file. The main code consists of two (2) concepts implemented in \Python: an example \Python-configuration file that is meant to be user configurable and the \onsubpy\ script that executes commands as dictated by command line arguments and the configuration file.

\tableofcontents

\section{Summary}

All provided files, with a short description, are:
\begin{itemize}
\item \onsubpy\ ~~~~~~~~~ --- the main execution code (<500 lines)
\item \lstinline{onsub} ~~~~~~~~~~~~ --- very short \Python\ script that calls \onsubpy\
\item \lstinline{pushd.py} ~~~~~~~~~ --- implements shell-like pushd capability (borrowed)
\item \lstinline{onsubbuiltin.py} ~~ --- config file that implements very basic \hg, \git, \svn\ behavior
\item \lstinline{onsubdefaults.py} ~ --- sample configuration file that just loads \lstinline{~/.onsublocal.py}
\item \lstinline{onsubexample.py} ~~ --- sample configuration file
\item \lstinline{onsubcheck.py} ~~~~ --- sample configuration file for complex examples
\item \lstinline{guestrepo2onsub.py} --- converter from guestrepo to onsub syntax
\item \lstinline{onsub2file.py} ~~~~ --- prints to stdout a file listing of prioritized folders
\item \lstinline{onsub2revs.py} ~~~~ --- prints to stdout a file listing of prioritized folders with
\item \lstinline{gitcheck.py} ~~~~~~ --- checks status of \git\ clone with recommended commands
\item \lstinline{hgcheck.py} ~~~~~~~ --- checks status of \hg\ clone with recommended commands
\item \lstinline{svncheck.py} ~~~~~~ --- checks status of \svn\ clone with recommended commands
\item \lstinline{README.md} ~~~~~~~~ --- this file
\end{itemize}

The configuration file is described later but it is organized into sections and provides rules for operation. The \lstinline{onsub} command can be run in two main modes: file mode and recurse mode.

The first mode of operation is file mode, where a special file is provided indicating which subfolders are to be visited and, optionally, information about how to create them if they are missing. File mode is indicated by the \lstinline{--file FILE} command line argument. The \lstinline{FILE} parameter indicates the input file of expected folders, grouped by section (described below) and, optionally, additional arguments for each expected folder. If the \lstinline{--make} command line argument is provided, then this mode can be used to construct all of the folders described by \lstinline{FILE} in parallel according to rules for the corresponding section in the configuration file.  Example:

\begin{snugshade}
\begin{lstlisting}[language=python]
git = [
		("./onsub", "https://bitbucket.org/sawolford/onsub.git"),
]
\end{lstlisting}
\end{snugshade}

The second mode of operation is recurse mode, where the filesystem is recursively searched, infinitely or up to a given depth. The configuration code is then queried for each folder in order to see if the \lstinline{onsub} command should execute a command in that folder and how to do so. If the provided command indicates that variable substitution is necessary, the configuration file can also specify rules for how the command is constructed.

Recurse mode is indicated by the lack of a \lstinline{--file FILE} command line argument. Instead of iterating over provided folder names, \lstinline{onsub} instead recurses through the file system to visit folders. Example:

\begin{snugshade}
\begin{lstlisting}[language=bash]
$ onsub pwd
\end{lstlisting}
\end{snugshade}

Both modes will execute commands according to the provided command line arguments. Extensive examples are provided at the end but some simple examples capture the main ways that commands are provided on the \lstinline{onsub} command line. Examples:

\begin{snugshade}
\begin{lstlisting}[language=bash]
$ onsub --file input.py --make          # constructs folders from input.py in parallel
$ onsub --file input.py --make {remote} # constructs folders from input.py in parallel, then executes "{remote}" command in each
$ onsub {cmd} status                    # recurses to all subfolders and runs "{cmd} status"
$ onsub pwd                             # recurses to all subfolders and runs "pwd"
\end{lstlisting}
\end{snugshade}

\section{Configuration}
The configuration file for \lstinline{onsub} is nothing but a \Python\ script, so it follows normal \Python\ syntax rules. The code is interpreted by \lstinline{onsub} as a color configuration list (\lstinline{colors}), a default argument list (\lstinline{arguments}), and a set of \Python\ dictionaries, each of which is known as a section. The \lstinline{colors} list instructs \lstinline{onsub} how to colorize output. The \lstinline{arguments} list contains arguments that are always passed to \lstinline{onsub}. Each section is just a normal \Python\ dictionary with keys and values. Sections are optional but, in the absence of extensive command line arguments, essentially required for onsub to do much more than trivial work.

The section names and most of the contents are arbitrary but are interpreted by \lstinline{onsub} in such a way that work may be performed on file folders. Most dictionary keys in a section entry are completely arbitrary and will be passed to the \Python\ \lstinline{string.format()} function in order to construct shell commands. This formatting process is repeated a set number of times or until the string no longer changes. If the value is a \Python\ function, then it is called to create the substitution string. If a key is prefixed by \lstinline{py:}, then the value is instead interpreted as a \Python\ function whose arguments are prescribed for certain keys (described below) or just a list of remaining command line arguments.

\subsection{Special dictionary keys}

\subsubsection{\lstinline{py:enable}}

Must evaluate directly to \lstinline{True} or \lstinline{False} (default: \lstinline{False}). Indicates that the section is to be interpreted directly by \lstinline{onsub}. Setting this to \lstinline{False} can be used to construct a composite section that is then enabled itself. Example:

\begin{snugshade}
\begin{lstlisting}[language=python]
git["py:enable"] = True
\end{lstlisting}
\end{snugshade}

\subsubsection{\lstinline{py:priority}}

\Python\ function taking three (3) arguments: \lstinline{verbose}, \lstinline{debug}, \lstinline{path}. The first two are flags that can be used to control output. The last is the path that should be checked to see if the section applies. The current working directory context for this function call is also set to \lstinline{path}. The function should return zero if the section does not apply and return a non-zero priority if the section applies. Required for any enabled section. Example:

\begin{snugshade}
\begin{lstlisting}[language=python]
def gitpriority(verbose, debug, path): return 4 if os.path.exists(".git") else 0
\end{lstlisting}
\end{snugshade}

\subsubsection{\lstinline{py:makecommand}}

\Python\ function taking four (4) arguments: \lstinline{verbose}, \lstinline{debug}, \lstinline{path}, \lstinline{*rest}. The first two are flags that can be used to control output. The third is the path that does not exist or needs to be updated. The last is a list for accepting a variable number of arguments. These variable arguments are taken from an input file (described later) and should typically contain additional instructions for constructing a missing folder. The function should return a string that evaluates to a shell command. Required if construction is requested (\lstinline{--make}) and \lstinline{py:makefunction} is not set (\lstinline{py:makecommand} takes precedence over \lstinline{py:makefunction}). Example:

\begin{snugshade}
\begin{lstlisting}[language=python]
def gitmakecommand(verbose, debug, path, *rest):
	if os.path.exists(path): return None
	assert len(rest) >= 1
	url = rest[0]
	cmd = "{{cmd}} clone {url} {path}".format(url=url, path=path)
	if debug: print(cmd)
	return cmd
\end{lstlisting}
\end{snugshade}

\subsubsection{\lstinline{py:makefunction}}

\Python\ function taking five (5) arguments: \lstinline{verbose}, \lstinline{debug}, \lstinline{path}, \lstinline{noexec}, \lstinline{*rest}. The first two are flags that can be used to control output. The third is the path that does not exist or needs to be updated. The fourth is a flag indicating that the function should not actually execute. The last is a list for accepting a variable number of arguments. These variable arguments are taken from an input file (described later) and should typically contain additional instructions for constructing a missing folder. The function should actually perform the necessary operation to construct a missing folder and should return a tuple consisting of an error code and an output string. This command will not be executed in parallel due to limitations of \Python\ multiprocessing. Required if construction is requested and \lstinline{py:makecommand} is not set. Example:

\begin{snugshade}
\begin{lstlisting}[language=python]
def allmakefunction(verbose, debug, path, noexec, *rest):
	if os.path.exists(path): return 0, 'path "{path}" already exists'.format(path=path)
	if verbose >=4: print("os.makedirs({path})".format(path=path))
	if noexec: return 0, "[noexec] os.makedirs({path})".format(path=path)
	os.makedirs(path)
	return 0, "os.makedirs({path})".format(path=path)
\end{lstlisting}
\end{snugshade}

\subsection{Substition types}

\subsubsection{\lstinline{string.format()} strings}

Some dictionary entries are strings that can contain \Python\ variable substitution specifiers compatible with the \Python\ \lstinline{string.format()} standard method. Variables are substituted from the same section dictionary. This variable substitution is performed iteratively a set number of times or until the string no longer changes. Example:

\begin{snugshade}
\begin{lstlisting}[language=python]
"cmd": "git"
"-v": "-v"
"remote": "{cmd} remote {-v}"
\end{lstlisting}
\end{snugshade}

\subsubsection{\Python\ substitution generator}
Some dictionary entries are \Python\ functions that generate strings for variable substitution compatible with the \Python\ \lstinline{string.format()} standard method. Example:

\begin{snugshade}
\begin{lstlisting}[language=python]
def getcwd(path):
	try:
		with pd.pushd(path): return os.getcwd()
		pass
	except FileNotFoundError: pass
	return ""

defdefault = {
	"cwd": lambda v, d, path: getcwd(path),
}
\end{lstlisting}
\end{snugshade}

\subsection{Final configuration}
Since the configuration file is just normal \Python\ code, complex configurations can be constructed. \Python\ allows a dictionary to be updated from another dictionary, where only keys in the other dictionary are replaced in the original. Checks against operating system type can be performed to alter behavior depending on the underlying OS. The resultant dictionary, if enabled, is the only one that will be interpreted by \lstinline{onsub}. Example:

\begin{snugshade}
\begin{lstlisting}[language=python]
git = {}
git.update(default)
git.update(gitdefault)
if os.name == "nt": git.update(gitwindows)
else: git.update(gitlinux)
git["py:enable"] = True
\end{lstlisting}
\end{snugshade}

The default configuration file is located at \lstinline{~/.onsub.py}. Some very simple fallback configurations are loaded from the \href{https://bitbucket.org/sawolford/onsub/src/master/config/onsubbuiltin.py}{builtin configuration file} (\lstinline{onsubbuiltin.py}). An extremely limited example file is provided by the \href{https://bitbucket.org/sawolford/onsub/src/master/config/onsubexample.py}{example configuration file}  (\lstinline{onsubexample.py}). A much more sophisticated functional \href{https://bitbucket.org/sawolford/onsub/src/master/config/onsubcheck.py}{example configuration file} (\lstinline{onsubcheck.py}) provides sample commands for \href{https://git-scm.com/}{\Git}, \href{https://www.mercurial-scm.org/}{\Mercurial} and \href{https://subversion.apache.org/}{\Subversion}.

\subsection{Details (using \lstinline{onsubbuiltin.py})}

\subsubsection{Pseudo-section \lstinline{colors}}

Colors to use in colorized output. Commented lines indicate default values for each type of output.

Value:

\begin{snugshade}
\begin{lstlisting}[language=python]
colors = {
	"path": Fore.BLUE + Back.RESET + Style.BRIGHT,
	"command": Fore.CYAN + Back.RESET + Style.BRIGHT,
	"errorcode": Fore.RED + Back.RESET + Style.BRIGHT,
	"partition": Fore.YELLOW + Back.RESET + Style.BRIGHT,
	# "good": Fore.GREEN + Back.RESET + Style.BRIGHT,
	# "bad": Fore.MAGENTA + Back.RESET + Style.BRIGHT,
	# "error": Fore.RED + Back.RESET + Style.BRIGHT,
}
\end{lstlisting}
\end{snugshade}

Explanation:

\begin{itemize}
\item \lstinline{"path"} ~~~~ -- Folder where command is executed
\item \lstinline{"command"} ~ -- Command that is executed
\item \lstinline{"errorcode"} -- Command exit code
\item \lstinline{"partition"} -- Separator between normal output and repeated error output
\item \lstinline{"good"} ~~~~ -- Output, if error code of command is zero (default: not used)
\item \lstinline{"bad"} ~~~~~ -- Output, if error code of command is non-zero (default: not used)
\item \lstinline{"error"} ~~~ -- Final output repeated for those commands with non-zero error code (default: not used)
\end{itemize}

\subsubsection{Pseudo-section \lstinline{arguments}}
Allows the same arguments to be passed to all invocations of \lstinline{onsub}.

Value:

\begin{snugshade}
\begin{lstlisting}[language=python]
arguments = [
	# "--chdir", ".",
	# "--color",
	# "--comment", "ignored",
	# "--count", "10",
	# "--debug",
	# "--depth", "1",
	# "--disable", "all",
	# "--discard",
	# "--dump",
	# "--dumpall",
	# "--enable", "all",
	# "--file", "subs.py",
	# "--hashed",
	# "--ignore", ".hg", "--ignore", ".git", "--ignore", ".svn",
	# "--invert",
	# "--make",
	# "--nocolor",
	# "--noenable",
	# "--noexec",
	# "--nofile",
	# "--nohashed",
	# "--noignore",
	# "--nomake",
	# "--norecurse",
	# "--preconfig", "",
	# "--postconfig", "",
	# "--py:closebrace", "%]",
	# "--py:enable", "py:enable",
	# "--py:makecommand", "py:makecommand",
	# "--py:makefunction", "py:makefunction",
	# "--py:openbrace", "%[",
	# "--py:priority", "py:priority",
	# "--recurse",
	# "--sleepmake", ".1",
	# "--sleepcommand", "0",
	# "--suppress",
	# "--verbose", "5",
	# "--workers", "8",
]
\end{lstlisting}
\end{snugshade}

\subsubsection{Pseudo-section \lstinline{default}}

The example \lstinline{default} section provides some sample \Linux\ commands and a sample \Python\ operation function.

\subsubsection*{\lstinline{default} summary}

\begin{itemize}
\item \lstinline{getcwd} ~~~ -- \Python\ helper function that returns the current working directory
\item \lstinline{defdefault} -- Pseudo-section used by default for all OSs
\item \lstinline{deflinux} ~ -- Pseudo-section used by default for \Linux
\item \lstinline{defwindows} -- Pseudo-section used by default for \Windows
\item \lstinline{default} ~~ -- Pseudo-section used by later sections
\end{itemize}

\subsubsection*{\lstinline{default} \Linux}

Composite:

\begin{snugshade}
\begin{lstlisting}[language=python]
default = {
	onsub = onsub
	cwd = <function <lambda> at 0x1079d5c20>
	sep = ;
	echo = /bin/echo
}
\end{lstlisting}
\end{snugshade}

Explanation:

\begin{itemize}
\item \lstinline{onsub} -- Self-reference for \lstinline{onsub} program
\item \lstinline{cwd} ~ -- \Python\ function that returns the current working directory
\item \lstinline{sep} ~ -- Platform-specific separator to use between shell commands
\item \lstinline{echo}~ -- Path to \lstinline{echo} command
\end{itemize}

\subsubsection*{\lstinline{default} details}

\begin{snugshade}
\begin{lstlisting}[language=python]
def getcwd(path):
	try:
		with pd.pushd(path): return os.getcwd()
		pass
	except FileNotFoundError: pass
	return ""
	
defdefault = {
	"onsub": "onsub",
	"cwd": lambda v, d, path: getcwd(path),
}
deflinux = {
	"sep": ";",
	"echo": "/bin/echo",
}
defwindows = { "sep": "&", }
	
default = {}
default.update(defdefault)
if os.name =="nt": default.update(defwindows)
else: default.update(deflinux)
\end{lstlisting}
\end{snugshade}

\subsubsection{Section \git}
The example \git\ section provides sample \git\ commands.

\subsubsection*{\git\ summary}

\begin{itemize}
\item \lstinline{gitmakecommand} -- \Python\ helper function that makes or updates a \git\ clone
\item \lstinline{gitpriority} ~~ -- \Python\ helper function that checks if folder is a \git\ folder
\item \lstinline{gitdefault} ~~~ -- Pseudo-section for all OSs for \git
\item \lstinline{gitlinux} ~~~~~ -- Pseudo-section for all \Linux\ for \git
\item \lstinline{gitwindows} ~~~ -- Pseudo-section for all \Windows\ for \git
\item \git\ ~~~~~~~~~~ -- Configuration section for \git
\end{itemize}

\subsubsection*{\git\ \Linux}

Composite:

\begin{snugshade}
\begin{lstlisting}[language=python]
git = {
	onsub = onsub
	cwd = <function <lambda> at 0x10db01c20>
	sep = ;
	echo = /bin/echo
	type = echo '(git)' {cwd}
	ctype = echo '(git)' {cwd}
	py:priority = <function gitpriority at 0x10db01d40>
	py:makecommand = <function gitsubsmakecommand at 0x10db30830>
	cmd = git
	remote = {cmd} remote get-url origin
	allremote = {cmd} remote -v
	wcrev = {cmd} rev-parse --verify --short HEAD
	py:enable = True
}
\end{lstlisting}
\end{snugshade}

Explanation:

\begin{itemize}
\item \lstinline{onsub} ~~~~~~~~ -- Inherited from default pseudo-section
\item \lstinline{cwd} ~~~~~~~~~~ -- Inherited from default pseudo-section
\item \lstinline{sep} ~~~~~~~~~~ -- Inherited from default pseudo-section
\item \lstinline{echo} ~~~~~~~~~ -- Inherited from default pseudo-section
\item \lstinline{type} ~~~~~~~~~ -- Command alias
\item \lstinline{ctype} ~~~~~~~~ -- Command alias
\item \lstinline{py:priority} ~~ -- \Python\ function that establishes a folder applies to this section
\item \lstinline{py:makecommand} -- \Python\ function that returns a shell command for cloning \git\ folder
\item \lstinline{cmd} ~~~~~~~~~~ -- Command alias
\item \lstinline{wcrev} ~~~~~~~~ -- Command alias
\item \lstinline{remote} ~~~~~~~ -- Command alias
\item \lstinline{allremote} ~~~~ -- Command alias
\item \lstinline{py:enable} ~~~~ -- \Python\ flag indicating that this section is enabled by default
\end{itemize}

\subsubsection*{\git\ details}

\begin{snugshade}
\begin{lstlisting}[language=python]
def gitmakecommand(verbose, debug, path, *rest):
	if os.path.exists(path): return None
	assert len(rest) >= 1
	url = rest[0]
	cmd = "{{cmd}} clone {url} {path}".format(url=url, path=path)
	if debug: print(cmd)
	return cmd

def gitpriority(verbose, debug, path): return 4 if os.path.exists(".git") else 0

gitdefault = {
	"type": "echo '(git)' {cwd}",
	"ctype": "echo " + Fore.GREEN + Back.RESET + Style.BRIGHT + "'(git)'" + Fore.RESET + " {cwd}",
	"py:priority": gitpriority,
	"py:makecommand": gitmakecommand,
	"cmd": "git",
	"remote": "{cmd} remote get-url origin",
	"allremote": "{cmd} remote -v",
	"wcrev": "{cmd} rev-parse --verify --short HEAD",
}
gitlinux = {}
gitwindows = {}

git = {}
git.update(default)
git.update(gitdefault)
if os.name == "nt": git.update(gitwindows)
else: git.update(gitlinux)
git["py:enable"] = True
\end{lstlisting}
\end{snugshade}

\subsubsection{Section \hg}

The example \hg\ section provides sample \hg\ commands.

\subsubsection*{\hg\ summary}

\begin{itemize}
\item \lstinline{hgmakecommand} -- \Python\ helper function that makes an \hg\ clone
\item \lstinline{hgpriority} ~~ -- \Python\ helper function checks if folder is a \hg\ folder
\item \lstinline{hgdefault} ~~~ -- Pseudo-section for all OSs for \hg
\item \lstinline{hglinux} ~~~~~ -- Pseudo-section for all \Linux\ for \hg
\item \lstinline{hgwindows} ~~~ -- Pseudo-section for all \Windows\ for \hg
\item \hg\ ~~~~~~~~~~ -- Configuration section for \hg
\end{itemize}

\subsubsection*{\hg\ \Linux}

Composite:

\begin{snugshade}
\begin{lstlisting}[language=python]
hg = {
	onsub = onsub
	cwd = <function <lambda> at 0x1025bfc20>
	sep = ;
	echo = /bin/echo
	type = echo '(hg)' {cwd}
	ctype = echo '(hg)' {cwd}
	py:priority = <function hgpriority at 0x1025bfe60>
	py:makecommand = <function hgsubsmakecommand at 0x1025ee8c0>
	cmd = hg
	wcrev = {cmd} id -i
	remote = {cmd} paths default
	allremote = {echo} -n "default = "; {cmd} paths default; {echo} -n "default-push = "; {cmd} paths default-push; {echo} -n "default-pull = "; {cmd} paths default-pull
	py:enable = True
}
\end{lstlisting}
\end{snugshade}

Explanation:

\begin{itemize}
\item \lstinline{onsub} ~~~~~~~~ -- Inherited from default pseudo-section
\item \lstinline{cwd} ~~~~~~~~~~ -- Inherited from default pseudo-section
\item \lstinline{sep} ~~~~~~~~~~ -- Inherited from default pseudo-section
\item \lstinline{echo} ~~~~~~~~~ -- Inherited from default pseudo-section
\item \lstinline{type} ~~~~~~~~~ -- Command alias
\item \lstinline{ctype} ~~~~~~~~ -- Command alias
\item \lstinline{py:priority} ~~ -- \Python\ function that establishes a folder applies to this section
\item \lstinline{py:makecommand} -- \Python\ function that returns a shell command for cloning \hg\ folder
\item \lstinline{cmd} ~~~~~~~~~~ -- Command alias
\item \lstinline{wcrev} ~~~~~~~~ -- Command alias
\item \lstinline{remote} ~~~~~~~ -- Command alias
\item \lstinline{allremote} ~~~~ -- Command alias
\item \lstinline{py:enable} ~~~~ -- \Python\ flag indicating that this section is enabled by default
\end{itemize}

\subsubsection*{\hg\ details}

\begin{snugshade}
\begin{lstlisting}[language=python]
def hgmakecommand(verbose, debug, path, *rest):
	if os.path.exists(path): return None
	assert len(rest) >= 1
	rrev = ""
	if len(rest) >= 2: rrev = "-r {rev}".format(rev=rest[1])
	url = rest[0]
	cmd = "{{cmd}} clone {url} {path} {rrev}".format(url=url, path=path, rrev=rrev)
	if debug: print(cmd)
	return cmd
	
def hgpriority(verbose, debug, path): return 3 if os.path.exists(".hg") else 0
	
hgdefault =  {
	"type": "echo '(hg)' {cwd}",
	"ctype": "echo " + Fore.CYAN + Back.RESET + Style.BRIGHT + "'(hg)'" + Fore.RESET + " {cwd}",
	"py:priority": hgpriority,
	"py:makecommand": hgmakecommand,
	"cmd": "hg",
	"wcrev": "{cmd} id -i",
}
hglinux = {
	"remote": '{cmd} paths default',
	"allremote": '{echo} -n "default = "; {cmd} paths default; {echo} -n "default-push = "; {cmd} paths default-push; {echo} -n "default-pull = "; {cmd} paths default-pull',
}
hgwindows = {
	"remote": '{cmd} paths default',
	"allremote": '{remote}; {cmd} paths default-push; {cmd} paths default-pull',
}
	
hg = {}
hg.update(default)
hg.update(hgdefault)
if os.name == "nt": hg.update(hgwindows)
else: hg.update(hglinux)
hg["py:enable"] = True
\end{lstlisting}
\end{snugshade}

\subsubsection{Section \svn}

The example \svn\ section provides some sample \Linux\ commands and a sample \Python\ operation function.

\subsubsection*{\svn\ summary}

\begin{itemize}
\item \lstinline{svnmakecommand} -- \Python\ helper function that makes a \svn\ checkout
\item \lstinline{svnpriority} ~~ -- \Python\ helper function checks if folder is a \svn\ folder
\item \lstinline{svndefault} ~~~ -- Pseudo-section for all OSs for \svn
\item \lstinline{svnlinux} ~~~~~ -- Pseudo-section for all \Linux\ for \svn
\item \lstinline{svnwindows} ~~~ -- Pseudo-section for all \Windows\ for \svn
\item \svn\ ~~~~~~~~~~ -- Configuration section for \svn
\end{itemize}

\subsubsection*{\svn\ \Linux}

Composite:

\begin{snugshade}
\begin{lstlisting}[language=python]
svn = {
	onsub = onsub
	cwd = <function <lambda> at 0x10e7f7c20>
	sep = ;
	echo = /bin/echo
	type = echo '(svn)' {cwd}
	ctype = echo '(svn)' {cwd}
	py:priority = <function svnpriority at 0x10e7f7f80>
	py:makecommand = <function svnsubsmakecommand at 0x10e826950>
	cmd = svn
	remote = {cmd} info --show-item url
	allremote = {remote}
	wcrev = {cmd} info --show-item revision
	py:enable = True
}
\end{lstlisting}
\end{snugshade}

Explanation:

\begin{itemize}
\item \lstinline{onsub} ~~~~~~~~ -- Inherited from default pseudo-section
\item \lstinline{cwd} ~~~~~~~~~~ -- Inherited from default pseudo-section
\item \lstinline{sep} ~~~~~~~~~~ -- Inherited from default pseudo-section
\item \lstinline{echo} ~~~~~~~~~ -- Inherited from default pseudo-section
\item \lstinline{type} ~~~~~~~~~ -- Command alias
\item \lstinline{ctype} ~~~~~~~~ -- Command alias
\item \lstinline{py:priority} ~~ -- \Python\ function that establishes a folder applies to this section
\item \lstinline{py:makecommand} -- \Python\ function that returns a shell command for checking out \svn\ folder
\item \lstinline{cmd} ~~~~~~~~~~ -- Command alias
\item \lstinline{wcrev} ~~~~~~~~ -- Command alias
\item \lstinline{remote} ~~~~~~~ -- Command alias
\item \lstinline{allremote} ~~~~ -- Command alias
\item \lstinline{py:enable} ~~~~ -- \Python\ flag indicating that this section is enabled by default
\end{itemize}

\subsubsection*{\svn\ details}

\begin{snugshade}
\begin{lstlisting}[language=python]
def svnmakecommand(verbose, debug, path, *rest):
	if os.path.exists(path): return None
	assert len(rest) >= 1
	rev = "head"
	if len(rest) >= 2: rev = rest[1]
	url = rest[0]
	cmd = "{{cmd}} checkout {url}@{rev} {path}".format(url=url, path=path, rev=rev)
	if debug: print(cmd)
	return cmd
	
def svnpriority(verbose, debug, path): return 2 if os.path.exists(".svn") else 0
	
svndefault = {
	"type": "echo '(svn)' {cwd}",
	"ctype": "echo " + Fore.MAGENTA + Back.RESET + Style.BRIGHT + "'(svn)'" + Fore.RESET + " {cwd}",
	"py:priority": svnpriority,
	"py:makecommand": svnmakecommand,
	"cmd": "svn",
	"remote": "{cmd} info --show-item url",
	"allremote": "{remote}",
	"wcrev": "{cmd} info --show-item revision"
}
svnlinux = {}
svnwindows = {}
	
svn = {}
svn.update(default)
svn.update(svndefault)
if os.name == "nt": svn.update(svnwindows)
else: svn.update(svnlinux)
svn["py:enable"] = True
\end{lstlisting}
\end{snugshade}

\subsubsection{Section \lstinline{all}}

The example \lstinline{all} section provides some sample \Linux\ commands and a sample \Python\ operation function.

\subsubsection*{\lstinline{all} summary}

\begin{itemize}
\item \lstinline{allmakecommand} -- \Python\ helper function that makes a folder
\item \lstinline{allpriority} ~~ -- \Python\ helper function that returns True
\item \lstinline{alldefault} ~~~ -- Pseudo-section for all OSs
\item \lstinline{alllinux} ~~~~~ -- Pseudo-section for all \Linux
\item \lstinline{allwindows} ~~~ -- Pseudo-section for all \Windows
\item \lstinline{all} ~~~~~~~~~~ -- Configuration section for \lstinline{all} folders
\end{itemize}

\subsubsection*{\lstinline{all} \Linux}

Composite:

\begin{snugshade}
\begin{lstlisting}[language=python]
all = {
	onsub = onsub
	cwd = <function <lambda> at 0x109e36c20>
	sep = ;
	echo = /bin/echo
	type = echo '(all)' {cwd}
	ctype = {type}
	py:priority = <function allpriority at 0x109e3c0e0>
	py:makefunction = <function allmakefunction at 0x109e3c050>
}
\end{lstlisting}
\end{snugshade}

Explanation:

\begin{itemize}
\item \lstinline{onsub} ~~~~~~~~~ -- Inherited from default pseudo-section
\item \lstinline{cwd} ~~~~~~~~~~~ -- Inherited from default pseudo-section
\item \lstinline{sep} ~~~~~~~~~~~ -- Inherited from default pseudo-section
\item \lstinline{echo} ~~~~~~~~~~ -- Inherited from default pseudo-section
\item \lstinline{cwd} ~~~~~~~~~~~ -- Inherited from default pseudo-section
\item \lstinline{type} ~~~~~~~~~~ -- Command alias
\item \lstinline{ctype} ~~~~~~~~~ -- Command alias
\item \lstinline{py:priority} ~~~ -- \Python\ function that establishes a folder applies to this section
\item \lstinline{py:makefunction} -- \Python\ function that makes a folder
\end{itemize}

\subsubsection*{\lstinline{all} details}

\begin{snugshade}
\begin{lstlisting}[language=python]
def allmakefunction(verbose, debug, path, noexec, *rest):
	if os.path.exists(path): return 0, 'path "{path}" already exists'.format(path=path)
	if verbose >=4: print("os.makedirs({path})".format(path=path))
	if noexec: return 0, "[noexec] os.makedirs({path})".format(path=path)
	os.makedirs(path)
	return 0, "os.makedirs({path})".format(path=path)
	
def allpriority(verbose, debug, path): return 1
	
alldefault = {
	"type": "echo '(all)' {cwd}",
	"ctype": "{type}",
	"py:priority": allpriority,
	"py:makefunction": allmakefunction,
}
alllinux = {}
allwindows = {}
	
all = {}
all.update(default)
all.update(alldefault)
if os.name == "nt": all.update(allwindows)
else: all.update(alllinux)
\end{lstlisting}
\end{snugshade}

\section{Command line options}

The basic command line options are:

\begin{snugshade}
\begin{lstlisting}[language=bash]
usage: onsub [-h] [--chdir CHDIR] [--color] [--comment COMMENT] [--configfile CONFIGFILE] [--count COUNT] [--debug]
			 [--depth DEPTH] [--disable DISABLE] [--discard] [--dump DUMP] [--dumpall] [--enable ENABLE] [--file FILE]
			 [--hashed] [--ignore IGNORE] [--invert] [--make] [--nocolor] [--noenable] [--noexec] [--nofile]
			 [--nohashed] [--noignore] [--nomake] [--norecurse] [--postconfig POSTCONFIG] [--preconfig PRECONFIG]
			 [--py:closebrace PYCLOSEBRACE] [--py:enable PYENABLE] [--py:makecommand PYMAKECOMMAND]
			 [--py:makefunction PYMAKEFUNCTION] [--py:openbrace PYOPENBRACE] [--py:priority PYPRIORITY] [--recurse]
			 [--sleepcommand SLEEPCOMMAND] [--sleepmake SLEEPMAKE] [--suppress] [--verbose VERBOSE] [--workers WORKERS]
			 ...

walks filesystem executing arbitrary commands

positional arguments:
  rest

optional arguments:
  -h, --help                        show this help message and exit
  --chdir CHDIR                     chdir first
  --color                           enables colorized output
  --comment COMMENT                 ignored
  --configfile CONFIGFILE           config file
  --count COUNT                     count for substitutions
  --debug                           debug flag
  --depth DEPTH                     walk depth
  --disable DISABLE                 disable section
  --discard                         discard error codes
  --dump DUMP                       dump section
  --dumpall                         dump all sections
  --enable ENABLE                   enable section
  --file FILE                       file with folder names
  --hashed                          check hashes of unknown files
  --ignore IGNORE                   ignore folder names
  --invert                          invert error codes
  --make                            make folders
  --nocolor                         disables colorized output
  --noenable                        no longer enable any sections
  --noexec                          do not actually execute
  --nofile                          ignore file options
  --nohashed                        do not check hashes of unknown files
  --noignore                        ignore ignore options
  --nomake                          do not make folders
  --norecurse                       do not recurse into subfolders
  --postconfig POSTCONFIG           postconfig option
  --preconfig PRECONFIG             preconfig option
  --py:closebrace PYCLOSEBRACE      key for py:closebrace
  --py:enable PYENABLE              key for py:enable
  --py:makecommand PYMAKECOMMAND    key for py:makecommand
  --py:makefunction PYMAKEFUNCTION  key for py:makefunction
  --py:openbrace PYOPENBRACE        key for py:openbrace
  --py:priority PYPRIORITY          key for py:priority
  --recurse                         recurse into subfolders
  --sleepcommand SLEEPCOMMAND       sleep between command calls
  --sleepmake SLEEPMAKE             sleep between make calls
  --suppress                        suppress repeated error output
  --verbose VERBOSE                 verbose level
  --workers WORKERS                 number of workers
\end{snugshade}

\subsubsection*{\lstinline{-h, --help}}
\begin{itemize}
\item Help: \lstinline{-h, --help                        show this help message and exit}
\item Type: Flag
\item Default: \lstinline{<none>}
\item Option: \lstinline{<none>}
\item Repeat: No
\item Outputs command line options (see above).
\end{itemize}

\subsubsection*{\lstinline{--chdir CHDIR}}
\begin{itemize}
\item Help: \lstinline{--chdir CHDIR                     chdir first}
\item Type: Option
\item Default: \lstinline{<none> }
\item Option \lstinline{CHDIR}
\item Repeat: Yes
\item Changes directory before execution.
\end{itemize}

\subsubsection*{\lstinline{--color}}
\begin{itemize}
\item Help: \lstinline{--color                           enables colorized output}
\item Type: Flag
\item Default: \lstinline{True}
\item Option: \lstinline{<none>}
\item Repeat: No
\item Enables colorized output.
\end{itemize}

\subsubsection*{\lstinline{--comment COMMENT}}
\begin{itemize}
\item Help: \lstinline{--comment COMMENT                 ignored}
\item Type: Option
\item Default: \lstinline{<none>}
\item Option: \lstinline{COMMENT}
\item Repeat: Yes
\item Ignored. Can be used to document scripted command lines.
\end{itemize}

\subsubsection*{\lstinline{--configfile CONFIGFILE}}
\begin{itemize}
\item Help: \lstinline{--configfile CONFIGFILE           config file}
\item Type: Option
\item Default: \lstinline{~/.onsub.py}
\item Option: \lstinline{CONFIGFILE}
\item Repeat: No
\item Names \lstinline{CONFIGFILE} to be the configuration file to be loaded.
\end{itemize}

\subsubsection*{\lstinline{--count COUNT}}
\begin{itemize}
\item Help: \lstinline{--count COUNT                     count for substitutions}
\item Type: Option
\item Default: \lstinline{10}
\item Option: \lstinline{COUNT}
\item Repeat: No
\item Indicates that \lstinline{COUNT} is the maximum number of times to recursively substitute configuration entries to prevent infinite recursion.
\end{itemize}

\subsubsection*{\lstinline{--debug}}
\begin{itemize}
\item Help: \lstinline{--debug                           debug flag}
\item Type: Flag
\item Default: \lstinline{False}
\item Option: \lstinline{<none>}
\item Repeat: No
\item Causes \lstinline{onsub} to output more diagnostic information. Is also passed to \lstinline{py:} functions where they occur in configuration entries.
\end{itemize}

\subsubsection*{\lstinline{--depth DEPTH}}
\begin{itemize}
\item Help: \lstinline{-depth DEPTH                     walk depth}
\item Type: Option
\item Default: \lstinline{-1}
\item Option: \lstinline{DEPTH}
\item Repeat: No
\item If greater than zero, limits the filesystem depth that is walked by \lstinline{onsub} to folder entries with up to \lstinline{DEPTH}-many subdirectories.
\end{itemize}

\subsubsection*{\lstinline{--disable DISABLE}}
\begin{itemize}
\item Help: \lstinline{--disable DISABLE                 disable section}
\item Type: Option
\item Default: \lstinline{None}
\item Option: \lstinline{DISABLE}
\item Repeat: Yes
\item Disables section \lstinline{DISABLE} that would otherwise be enabled by configuration file or command line option. Takes precedence over all other means of enabling sections.
\end{itemize}

\subsubsection*{\lstinline{--discard}}
\begin{itemize}
\item Help: \lstinline{--discard                         discard error codes}
\item Type: Flag
\item Default: \lstinline{False}
\item Option: \lstinline{<none>}
\item Repeat: No
\item Whether to discard error codes.
\end{itemize}

\subsubsection*{\lstinline{--dump DUMP}}
\begin{itemize}
\item Help: \lstinline{--dump DUMP                       dump section}
\item Type: Option
\item Default: \lstinline{None}
\item Option: \lstinline{DUMP}
\item Repeat: Yes
\item Outputs configuration section \lstinline{DUMP}. Can dump multiple configuration sections. Can be used to remember how command substitutions will be generated.
\end{itemize}

\subsubsection*{\lstinline{--dumpall}}
\begin{itemize}
\item Help: \lstinline{--dumpall                         dump all sections}
\item Type: Flag
\item Default: \lstinline{False}
\item Option: \lstinline{<none>}
\item Repeat: No
\item Dumps all configuration sections.
\end{itemize}

\subsubsection*{\lstinline{--enable ENABLE}}
\begin{itemize}
\item Help: \lstinline{--enable ENABLE                   enabled sections}
\item Type: Option
\item Default: \lstinline{None}
\item Option: \lstinline{ENABLE}
\item Repeat: Yes
\item Enables section \lstinline{ENABLE} that would otherwise be disabled by configuration file or command line option. Takes precedence over default disable and configuration file.
\end{itemize}

\subsubsection*{\lstinline{--file FILE}}
\begin{itemize}
\item Help: \lstinline{--file FILE                       file with folder names}
\item Type: Option
\item Default: \lstinline{None}
\item Option: \lstinline{FILE}
\item Repeat: Yes
\item Reads \lstinline{FILE} as list of folders to be operated on instead of recursively scanning filesystem. Missing folders will be generated by \lstinline{py:makecommand} or \lstinline{py:makefunction} if available and in that order.
\end{itemize}

\subsubsection*{\lstinline{--hashed}}
\begin{itemize}
\item Help: \lstinline{--hashed                          check hashes of unknown files}
\item Type: Flag
\item Default: \lstinline{True}
\item Option: \lstinline{<none>}
\item Repeat: No
\item Whether to check hashes of \lstinline{--file FILE} file parameters. Prevents untrusted code from executing by accident.
\end{itemize}

\subsubsection*{\lstinline{--ignore IGNORE}}
\begin{itemize}
\item Help: \lstinline{--ignore IGNORE                   ignore folder names}
\item Type: Option
\item Default: \lstinline{None}
\item Option: \lstinline{IGNORE}
\item Repeat: Yes
\item Sets folder names that will not be visited when recursively searching file system.
\end{itemize}

\subsubsection*{\lstinline{--invert}}
\begin{itemize}
\item Help: \lstinline{--invert                          invert error codes}
\item Type: Flag
\item Default: \lstinline{False}
\item Option: \lstinline{<none>}
\item Repeat: No
\item Whether to invert error codes.
\end{itemize}

\subsubsection*{\lstinline{--make}}
\begin{itemize}
\item Help: \lstinline{--make                            make folders}
\item Type: Flag
\item Default: \lstinline{False}
\item Option: \lstinline{<none>}
\item Repeat: No
\item Indicates that missing folders should be created. This can be useful when folders will need to be generated because of \lstinline{--file FILE} command line option.
\end{itemize}

\subsubsection*{\lstinline{--nocolor}}
\begin{itemize}
\item Help: \lstinline{--nocolor                         disables colorized output}
\item Type: Flag
\item Default: \lstinline{False}
\item Option: \lstinline{<none>}
\item Repeat: No
\item Turns off color in output.
\end{itemize}

\subsubsection*{\lstinline{--noenable}}
\begin{itemize}
\item Help: \lstinline{--noenable                        no longer enable any sections}
\item Type: Option
\item Default: \lstinline{False}
\item Option: \lstinline{<none>}
\item Repeat: No
\item Defaults all configuration sections to not be enabled. This can be changed later for each configuration section with other command line options.
\end{itemize}

\subsubsection*{\lstinline{--noexec}}
\begin{itemize}
\item Help: \lstinline{--noexec                          do not actually execute}
\item Type: Flag
\item Default: \lstinline{False}
\item Option: \lstinline{<none>}
\item Repeat: No
\item Causes commands to be printed to the console but not actually executed.
\end{itemize}

\subsubsection*{\lstinline{--nofile}}
\begin{itemize}
\item Help: \lstinline{--nofile                          ignore file options}
\item Type: Flag
\item Default: \lstinline{False}
\item Option: \lstinline{<none>}
\item Repeat: No
\item Disables reading files passed to \lstinline{--file FILE}. This can be useful if a default command line is used.
\end{itemize}

\subsubsection*{\lstinline{--nohashed}}
\begin{itemize}
\item Help: \lstinline{--nohashed                        do not check hashes of unknown files}
\item Type: Flag
\item Default: \lstinline{False}
\item Option: \lstinline{<none>}
\item Repeat: No
\item Disables checking of hashes of \lstinline{--file FILE} files.
\end{itemize}

\subsubsection*{\lstinline{--noignore}}
\begin{itemize}
\item Help: \lstinline{--noignore                        ignore ignore options}
\item Type: Flag
\item Default: \lstinline{False}
\item Option: \lstinline{<none>}
\item Repeat: No
\item Disables \lstinline{--ignore IGNORE} command line options. This can be useful if there's a default ignore in \lstinline{arguments}.
\end{itemize}

\subsubsection*{\lstinline{--nomake}}
\begin{itemize}
\item Help: \lstinline{--nomake                          do not make folders}
\item Type: Flag
\item Default: \lstinline{False}
\item Option: \lstinline{<none>}
\item Repeat: No
\item Disables \lstinline{--make} command line option. This can be useful if that arument is on by default in \lstinline{arguments}.
\end{itemize}

\subsubsection*{\lstinline{--norecurse}}
\begin{itemize}
\item Help: \lstinline{--norecurse                       do not recurse into subfolders}
\item Type: Flag
\item Default: \lstinline{False}
\item Option: \lstinline{<none>}
\item Repeat: No
\item Causes \lstinline{onsub} to recurse into subfolders.
\end{itemize}

\subsubsection*{\lstinline{--postconfig POSTCONFIG}}
\begin{itemize}
\item Help: \lstinline{--postconfig POSTCONFIG           postconfig option}
\item Type: Option
\item Default: \lstinline{None}
\item Option: \lstinline{POSTCONFIG}
\item Repeat: Yes
\item Appends \lstinline{POSTCONFIG} lines to configuration, one at a time. Can be used to alter configuration for a single command execution.
\end{itemize}

\subsubsection*{\lstinline{--preconfig PRECONFIG}}
\begin{itemize}
\item Help: \lstinline{--preconfig PRECONFIG             preconfig option}
\item Type: Option
\item Default: \lstinline{None}
\item Option: \lstinline{PRECONFIG}
\item Repeat: Yes
\item Prepends \lstinline{PRECONFIG} lines to configuration, one at a time. Can be used to alter configuration for a single command execution.
\end{itemize}

\subsubsection*{\lstinline{--py:closebrace PYCLOSEBRACE}}
\begin{itemize}
\item Help: \lstinline{--py:closebrace PYCLOSEBRACE      key for py:closebrace}
\item Type: Option
\item Default: \lstinline{%]}
\item Option: \lstinline{PYCLOSEBRACE}
\item Repeat: No
\item Sets the substitution string for a literal close brace.
\end{itemize}

\subsubsection*{\lstinline{--py:enable PYENABLE}}
\begin{itemize}
\item Help: \lstinline{--py:enable PYENABLE              key for py:enable}
\item Type: Option
\item Default: \lstinline{py:enable}
\item Option: \lstinline{PYENABLE}
\item Repeat: No
\item Names \lstinline{PYENABLE} as the key to look up in each configuration section for determining if the section is enabled by default.
\end{itemize}

\subsubsection*{\lstinline{--py:makecommand PYMAKECOMMAND}}
\begin{itemize}
\item Help: \lstinline{--py:makecommand PYMAKECOMMAND    key for py:makecommand}
\item Type: Option
\item Default: \lstinline{py:makecommand}
\item Option: \lstinline{PYMAKECOMMAND}
\item Repeat: No
\item Names \lstinline{PYMAKECOMMAND} as the key to look up in each configuration section for generating commands to make folders.
\end{itemize}

\subsubsection*{\lstinline{--py:makefunction PYMAKEFUNCTION}}
\begin{itemize}
\item Help: \lstinline{--py:makefunction PYMAKEFUNCTION  key for py:makefunction}
\item Type: Option
\item Default: \lstinline{py:makefunction}
\item Option: \lstinline{PYMAKEFUNCTION}
\item Repeat: No
\item Names \lstinline{PYMAKEFUNCTION} as the key to look up in each configuration section for executing \Python\ commands to make folders.
\end{itemize}

\subsubsection*{\lstinline{--py:openbrace PYOPENBRACE}}
\begin{itemize}
\item Help: \lstinline{--py:openbrace PYOPENBRACE        key for py:openbrace}
\item Type: Option
\item Default: \lstinline{%[}
\item Option: \lstinline{PYOPENBRACE}
\item Repeat: No
\item Sets the substitution string for a literal open brace.
\end{itemize}

\subsubsection*{\lstinline{--py:pypriority PYPRIORITY}}
\begin{itemize}
\item Help: \lstinline{--py:pypriority PYPRIORITY        key for py:priority}
\item Type: Option
\item Default: \lstinline{py:priority}
\item Option: \lstinline{PYPRIORITY}
\item Repeat: No
\item Names \lstinline{PYPRIORITY} as the key to look up in each configuration section for checking to see if a section applies to a folder.
\end{itemize}

\subsubsection*{\lstinline{--recurse}}
\begin{itemize}
\item Help: \lstinline{--recurse                         recurse into subfolders}
\item Type: Option
\item Default: True
\item Option: \lstinline{<none>}
\item Repeat: No
\item Names \lstinline{PYPRIORITY} as the key to look up in each configuration section for checking to see if a section applies to a folder.
\end{itemize}

\subsubsection*{\lstinline{--sleepcommand SLEEPCOMMAND}}
\begin{itemize}
\item Help: \lstinline{--sleepcommand SLEEPCOMMAND       sleep between command calls}
\item Type: Option
\item Default: \lstinline{0}
\item Option: \lstinline{SLEEPCOMMAND}
\item Repeat: No
\item Sets the sleep value to issue between recursive commands. Can be used to throttle server connections.
\end{itemize}

\subsubsection*{\lstinline{--sleepmake SLEEPMAKE}}
\begin{itemize}
\item Help: \lstinline{--sleepmake SLEEPMAKE             sleep between make calls}
\item Type: Option
\item Default: \lstinline{0.1}
\item Option: \lstinline{SLEEPMAKE}
\item Repeat: No
\item Sets the sleep value to issue between make folder commands. Can be used to throttle server connections.
\end{itemize}

\subsubsection*{\lstinline{--suppress}}
\begin{itemize}
\item Help: \lstinline{--suppress                        suppress repeated error output}
\item Type: Flag
\item Default: \lstinline{False}
\item Option: \lstinline{<none>}
\item Repeat: No
\item Indicates that summary information for errors should not be output at end of execution. Output will otherwise be included depending on \lstinline{--verbose} flag.
\end{itemize}

\subsubsection*{\lstinline{--verbose VERBOSE}}
\begin{itemize}
\item Help: \lstinline{--verbose VERBOSE                 verbose level}
\item Type: Option
\item Default: \lstinline{4}
\item Option: \lstinline{VERBOSE}
\item Repeat: No
\item Sets the level of output to be included in command execution where higher numbers include output of lower numbers:
\begin{itemize}
\item[] 0 = No output
\item[] 1 = Only errors, at end of parallel execution
\item[] 2 = Output of commands
\item[] 3 = Output path, applicable section, and command
\item[] 4 = Output one line for each command (at point of execution)
\item[] 5 = Output of commands as execution finishes
\item[] 6 = Output of make commands as make execution finishes
\end{itemize}
\end{itemize}

\subsubsection*{\lstinline{--workers WORKERS}}
\begin{itemize}
\item Help: \lstinline{--workers WORKERS                 number of workers}
\item Type: Option
\item Default: \lstinline{<number of cores>}
\item Option: \lstinline{WORKERS}
\item Repeat: No
\item Sets the number of simultaneous worker processes to use.
\end{itemize}

\section{Simple Examples}

Simple examples of using \lstinline{onsub} are below. It is executed on a \Mac\ and the directory structure is assumed to be the following:

\begin{snugshade}
\begin{lstlisting}[language=bash]
git/
git/.git
hg/
hg/.hg
normal/
svn/
svn/.svn
\end{lstlisting}
\end{snugshade}
	
Due to the limited quoting ability of the \Windows\ \lstinline{CMD.EXE} command shell, some of these examples are too sophisticated to run correctly in that environment.

\subsubsection*{No command line options}

With \lstinline{ }:

\begin{snugshade}
\begin{lstlisting}[language=bash]
$ onsub pwd
svn (svn) pwd
hg (hg) pwd
git (git) pwd
<<< RESULTS >>>
svn (svn) pwd
/private/tmp/sample/svn
hg (hg) pwd
/private/tmp/sample/hg
git (git) pwd
/private/tmp/sample/git
\end{lstlisting}
\end{snugshade}
	
Identifies \svn\ as a \Subversion\ folder, \hg\ as a \Mercurial\ folder, and \git\ as a \Git\ folder and executes \lstinline{pwd} in those folders.

\subsubsection*{Do not enable by default (\lstinline{--noenable})}

With \lstinline{--noenable}:

\begin{snugshade}
\begin{lstlisting}[language=bash]
$ onsub --noenable pwd
\end{lstlisting}
\end{snugshade}
	
No sections are enabled so no commands are executed.

\subsubsection*{Enable section (\lstinline{--enable ENABLE})}

With \lstinline{--noenable --enable git}:

\begin{snugshade}
\begin{lstlisting}[language=bash]
$ onsub --noenable --enable git pwd
git (git) pwd
<<< RESULTS >>>
git (git) pwd
/private/tmp/sample/git
\end{lstlisting}
\end{snugshade}

Identifies \git\ as a \Git\ folder and executes command. Only \git\ section is applied.

\subsubsection*{Disable section (\lstinline{--disable DISABLE})}

With \lstinline{--disable svn --disable hg}:

\begin{snugshade}
\begin{lstlisting}[language=bash]
$ onsub --disable svn --disable hg pwd
git (git) pwd
<<< RESULTS >>>
git (git) pwd
/private/tmp/sample/git
\end{lstlisting}
\end{snugshade}

Identifies \git\ as a \Git\ folder and executes command. Only \git\ section is applied.

\subsubsection*{Ignore folders (\lstinline{--ignore IGNORE [--ignore IGNORE ...]})}

With \lstinline{--noenable --enable all --ignore .git --ignore .hg --ignore .svn}:

\begin{snugshade}
\begin{lstlisting}[language=bash]
$ onsub --noenable --enable all --ignore .git --ignore .hg --ignore .svn pwd
. (all) pwd
svn (all) pwd
normal (all) pwd
hg (all) pwd
git (all) pwd
<<< RESULTS >>>
. (all) pwd
/tmp/sample
svn (all) pwd
/private/tmp/sample/svn
hg (all) pwd
/private/tmp/sample/hg
git (all) pwd
/private/tmp/sample/git
normal (all) pwd
/private/tmp/sample/normal
\end{lstlisting}
\end{snugshade}

Visits all folder without traversing into ignored folders to execute command.

\subsubsection*{Dump config (\lstinline{--dump DUMP [--dump DUMP ...]})}

With \lstinline{--dump default}:

\begin{snugshade}
\begin{lstlisting}[language=bash]	
$ onsub --dump default
default = {
	onsub = onsub
	cwd = <function <lambda> at 0x10bafec20>
	sep = ;
	echo = /bin/echo
}
\end{lstlisting}
\end{snugshade}

With \lstinline{--dump git}:

\begin{snugshade}
\begin{lstlisting}[language=bash]	
$ onsub --dump git
git = {
	onsub = onsub
	cwd = <function <lambda> at 0x10f916c20>
	sep = ;
	echo = /bin/echo
	type = echo '(git)' {cwd}
	ctype = echo '(git)' {cwd}
	py:priority = <function gitpriority at 0x10f916d40>
	py:makecommand = <function gitmakecommand at 0x10f916cb0>
	cmd = git
	remote = {cmd} remote get-url origin
	allremote = {cmd} remote -v
	wcrev = {cmd} rev-parse --verify --short HEAD
	py:enable = True
}
\end{lstlisting}
\end{snugshade}

With \lstinline{--dump hg}:

\begin{snugshade}
\begin{lstlisting}[language=bash]	
$ onsub --dump hg
hg = {
	onsub = onsub
	cwd = <function <lambda> at 0x10b0d7c20>
	sep = ;
	echo = /bin/echo
	type = echo '(hg)' {cwd}
	ctype = echo '(hg)' {cwd}
	py:priority = <function hgpriority at 0x10b0d7e60>
	py:makecommand = <function hgmakecommand at 0x10b0d7dd0>
	cmd = hg
	wcrev = {cmd} id -i
	remote = {cmd} paths default
	allremote = {echo} -n "default = "; {cmd} paths default; {echo} -n "default-push = "; {cmd} paths default-push; {echo} -n "default-pull = "; {cmd} paths default-pull
	py:enable = True
}
\end{lstlisting}
\end{snugshade}

With \lstinline{--dump svn}:

\begin{snugshade}
\begin{lstlisting}[language=bash]
$ onsub --dump svn
svn = {
	onsub = onsub
	cwd = <function <lambda> at 0x10cba5c20>
	sep = ;
	echo = /bin/echo
	type = echo '(svn)' {cwd}
	ctype = echo '(svn)' {cwd}
	py:priority = <function svnpriority at 0x10cba5f80>
	py:makecommand = <function svnmakecommand at 0x10cba5ef0>
	cmd = svn
	remote = {cmd} info --show-item url
	allremote = {remote}
	wcrev = {cmd} info --show-item revision
	py:enable = True
}
\end{lstlisting}
\end{snugshade}

With \lstinline{--dump all}:

\begin{snugshade}
\begin{lstlisting}[language=bash]	
$ onsub --dump all
all = {
	onsub = onsub
	cwd = <function <lambda> at 0x1013c8c20>
	sep = ;
	echo = /bin/echo
	type = echo '(all)' {cwd}
	ctype = {type}
	py:priority = <function allpriority at 0x1013cd0e0>
	py:makefunction = <function allmakefunction at 0x1013cd050>
}
\end{lstlisting}
\end{snugshade}

\subsubsection*{Limit recursion depth (\lstinline{--depth})}

With \lstinline{--enable all --depth 1}:

\begin{snugshade}
\begin{lstlisting}[language=bash]	
$ onsub --enable all --depth 1 pwd
. (all) pwd
<<< RESULTS >>>
. (all) pwd
/tmp/sample
\end{lstlisting}
\end{snugshade}

Only recurses one level deep.

With \lstinline{--enable all --depth 2}:

\begin{snugshade}
\begin{lstlisting}[language=bash]	
$ onsub --enable all --depth 2 pwd
. (all) pwd
svn (svn) pwd
normal (all) pwd
hg (hg) pwd
git (git) pwd
<<< RESULTS >>>
. (all) pwd
/tmp/sample
svn (svn) pwd
/private/tmp/sample/svn
git (git) pwd
/private/tmp/sample/git
normal (all) pwd
/private/tmp/sample/normal
hg (hg) pwd
/private/tmp/sample/hg
\end{lstlisting}
\end{snugshade}

Recurses an additional level into working copies.

With \lstinline{--enable all --depth 3}:

\begin{snugshade}
\begin{lstlisting}[language=bash]	
$ onsub --enable all --depth 3 pwd
. (all) pwd
svn (svn) pwd
svn/.svn (all) pwd
normal (all) pwd
hg (hg) pwd
hg/.hg (all) pwd
git (git) pwd
git/.git (all) pwd
<<< RESULTS >>>
. (all) pwd
/tmp/sample
svn (svn) pwd
/private/tmp/sample/svn
svn/.svn (all) pwd
/private/tmp/sample/svn/.svn
hg (hg) pwd
/private/tmp/sample/hg
normal (all) pwd
/private/tmp/sample/normal
hg/.hg (all) pwd
/private/tmp/sample/hg/.hg
git (git) pwd
/private/tmp/sample/git
git/.git (all) pwd
/private/tmp/sample/git/.git
\end{lstlisting}
\end{snugshade}

Recurses an additional level into working copy version control folders.

\subsubsection*{No execution (\lstinline{--noexec})}

With \lstinline{--noexec pwd}:

\begin{snugshade}
\begin{lstlisting}[language=bash]	
$ onsub --noexec pwd
svn (svn) pwd
hg (hg) pwd
git (git) pwd
<<< RESULTS >>>
svn (svn) pwd
[noexec] pwd
hg (hg) pwd
[noexec] pwd
git (git) pwd
[noexec] pwd
\end{lstlisting}
\end{snugshade}

Identifies \svn\ as a \Subversion\ folder, \hg\ as a \Mercurial\ folder, and \git\ as a \Git\ folder but only outputs the command that would have been executed.

\section{\Git\ examples}

Complex \Git\ examples of using \lstinline{onsub} are below. It is executed on a \Mac\ and the file \lstinline{input.py} is assumed to contain the following:

\begin{snugshade}
\begin{lstlisting}[language=python]	
git = [
	("./onsub1", "https://bitbucket.org/sawolford/onsub.git"),
	("./onsub2", "https://bitbucket.org/sawolford/onsub.git"),
]
\end{lstlisting}
\end{snugshade}

\subsection{Input file (\lstinline{--file FILE})}

With \lstinline{--file input.py --make}:

\begin{snugshade}
\begin{lstlisting}[language=bash]	
$ onsub --file input.py --make
./onsub1 (git) git clone https://bitbucket.org/sawolford/onsub.git ./onsub1
./onsub2 (git) git clone https://bitbucket.org/sawolford/onsub.git ./onsub2
<<< MAKE >>>
./onsub2 (git) git clone https://bitbucket.org/sawolford/onsub.git ./onsub2
Cloning into './onsub2'...
./onsub1 (git) git clone https://bitbucket.org/sawolford/onsub.git ./onsub1
Cloning into './onsub1'...
\end{lstlisting}
\end{snugshade}

Clones \lstinline{onsub} \Git\ repository locally.

\subsection{\Git\ fetch}

With \lstinline|--file input.py {cmd} fetch -v|

\begin{snugshade}
\begin{lstlisting}[language=bash]	
$ onsub --file input.py {cmd} fetch -v
onsub1 (git) git fetch -v
onsub2 (git) git fetch -v
<<< RESULTS >>>
onsub1 (git) git fetch -v
From https://bitbucket.org/sawolford/onsub
	= [up to date]      master     -> origin/master
onsub2 (git) git fetch -v
From https://bitbucket.org/sawolford/onsub
	= [up to date]      master     -> origin/master
\end{lstlisting}
\end{snugshade}

Fetches changesets and outputs current local branch and upstream branch.

\subsection{\Git\ pull}

With \lstinline|--file input.py {cmd} pull|

\begin{snugshade}
\begin{lstlisting}[language=bash]	
$ onsub --file input.py {cmd} pull
onsub1 (git) git pull
onsub2 (git) git pull
<<< RESULTS >>>
onsub2 (git) git pull
Already up to date.
onsub1 (git) git pull
Already up to date.
\end{lstlisting}
\end{snugshade}

\subsection{Selective folder \Git\ pull}

With \lstinline|--chdir onsub1 {cmd} pull|:

\begin{snugshade}
\begin{lstlisting}[language=bash]	
$ onsub --chdir onsub1 {cmd} pull
. (git) git pull
<<< RESULTS >>>
. (git) git pull
Already up to date.
\end{lstlisting}
\end{snugshade}

Changes directory to \lstinline{onsub1} and executes \lstinline{pull} command.

\section{Complex Examples}

Complex examples of using \lstinline{onsub} are below. They are executed on a \Mac\ and the configuration file is assumed to include \href{https://bitbucket.org/sawolford/onsub/src/master/config/onsubcheck.py}{onsubcheck.py}. Also, assume that the shell scripts \href{https://bitbucket.org/sawolford/onsub/src/master/scripts/gitcheck.py}{gitcheck}, \href{https://bitbucket.org/sawolford/onsub/src/master/scripts/hgcheck.py}{hgcheck}, \href{https://bitbucket.org/sawolford/onsub/src/master/scripts/svncheck.py}{svncheck} are on the shell command search path.

Prepare the sample folder with the following shell script:

\begin{snugshade}
\begin{lstlisting}[language=bash]	
#!/bin/sh
mkdir repo
mkdir wc
cd repo
svnadmin create svn
git init --bare git
hg init hg
cd ../wc
svn co file://$(realpath $(pwd)/../repo/svn) ./svn1
svn co file://$(realpath $(pwd)/../repo/svn) ./svn2
git clone ../repo/git ./git1
git clone ../repo/git ./git2
hg clone ../repo/hg ./hg1
hg clone ../repo/hg ./hg2
cd svn1
echo "svn1" > file.txt
svn add file.txt
svn ci -m "svn1"
echo "svn1 svn1" > file.txt
cd ../svn2
svn up
echo "svn1 svn2" > file.txt
cd ../git1
echo "git1" > file.txt
git add file.txt
git ci -a -m "git1"
git push
echo "git1 git1" > file.txt
cd ../git2
git pull
echo "git1 git2" > file.txt
git ci -a -m "git2"
echo "git1 git2 git2" > file.txt
cd ../hg1
echo "hg1" > file.txt
hg add file.txt
hg ci -m "hg1"
hg push
echo "hg1 hg1" > file.txt
cd ../hg2
hg pull
hg update
echo "hg1 hg2" > file.txt
hg ci -m "hg2"
echo "hg1 hg2 hg2" > file.txt
\end{lstlisting}
\end{snugshade}

At this point, there are "remote" repositorues (\lstinline{repo/git}, \lstinline{repo/hg}, \lstinline{repo/svn}), local clones with working copy changes (\lstinline{wc/git1}, \lstinline{wc/hg1}, \lstinline{wc/svn1}, \lstinline{wc/svn2}), and local clones with local commits (\lstinline{wc/git2}, \lstinline{wc/hg2}). The following example commands are run from the \lstinline{wc} subfolder.

\subsection{\Git}

\Git\ cannot check the remote server without fetching first but it does track the remote branches even after fetching. With those limitations and benefits, \lstinline{onsub} can easily support six (6) operations: \lstinline{get}, \lstinline{put}, \lstinline{download}, \lstinline{upload}, \lstinline{download-get}, \lstinline{put-upload}:

\begin{itemize}
\item \lstinline{get} ~~~~~~~~ -- merges with working copy (remote is unchanged)
\item \lstinline{put} ~~~~~~~~ -- commits local changes (remote is unchanged)
\item \lstinline{download} ~~~ -- fetches and rebases (working copy and remote are unchanged)
\item \lstinline{upload} ~~~~~ -- pushes local commits (working copy and local are unchanged)
\item \lstinline{download-get} -- fetches, rebases and merges with working copy (remote is unchanged)
\item \lstinline{put-upload} ~ -- commits local changes and pushes local commits to remote
\end{itemize}

Determine the current state of the \Git\ clones with \lstinline|{check}|:

\begin{snugshade}
\begin{lstlisting}[language=bash]	
$ onsub --disable hg --disable svn check
git2 (git) gitcheck
git1 (git) gitcheck
<<< RESULTS >>>
git1 (git) gitcheck
onsub --chdir /private/tmp/all/wc/git1 --depth 1 --comment "wc=1,sh=0,out=0,in=0" {put}
git2 (git) gitcheck
onsub --chdir /private/tmp/all/wc/git2 --depth 1 --comment "wc=1,sh=0,out=1,in=0" {put-upload}
<<< ERRORS >>>
(8) git1 (git) gitcheck
onsub --chdir /private/tmp/all/wc/git1 --depth 1 --comment "wc=1,sh=0,out=0,in=0" {put}
(4) git2 (git) gitcheck
onsub --chdir /private/tmp/all/wc/git2 --depth 1 --comment "wc=1,sh=0,out=1,in=0" {put-upload}
\end{lstlisting}
\end{snugshade}

Error \lstinline{8} indicates that the \lstinline{git1} folder has local changes to commit. The output command for that error instructs how those folders might be brought to a clean working copy and synced local clone.

We can execute the \lstinline{put} command to commit the working copy changes:

\begin{snugshade}
\begin{lstlisting}[language=bash]	
$ onsub --chdir /private/tmp/all/wc/git1 --depth 1 --comment "wc=1,sh=0,out=0,in=0" {put}
. (git) git commit -a -e -m "# $(pwd)"
<<< RESULTS >>>
. (git) git commit -a -e -m "# $(pwd)"
[master 6c1e32b] git1
	1 file changed, 1 insertion(+), 1 deletion(-)
	
$ onsub --disable hg --disable svncheck
git2 (git) gitcheck
git1 (git) gitcheck
<<< RESULTS >>>
git2 (git) gitcheck
onsub --chdir /private/tmp/all/wc/git2 --depth 1 --comment "wc=1,sh=0,out=1,in=0" {put-upload}
git1 (git) gitcheck
onsub --chdir /private/tmp/all/wc/git1 --depth 1 --comment "wc=0,sh=0,out=1,in=0" {upload}
<<< ERRORS >>>
(4) git2 (git) gitcheck
onsub --chdir /private/tmp/all/wc/git2 --depth 1 --comment "wc=1,sh=0,out=1,in=0" {put-upload}
(5) git1 (git) gitcheck
onsub --chdir /private/tmp/all/wc/git1 --depth 1 --comment "wc=0,sh=0,out=1,in=0" {upload}
\end{lstlisting}
\end{snugshade}

Error \lstinline{5} indicates that the \lstinline{git1} folder has commits to upload to the remote server. We can execute the \lstinline{upload} command to push those commits:

\begin{snugshade}
\begin{lstlisting}[language=bash]	
$ onsub --chdir /private/tmp/all/wc/git1 --depth 1 --comment "wc=0,sh=0,out=1,in=0" {upload}
. (git) git push
<<< RESULTS >>>
. (git) git push
To /tmp/all/wc/../repo/git
	278f253..6c1e32b  master -> master

$ onsub --disable hg --disable svn check
git2 (git) gitcheck
git1 (git) gitcheck
<<< RESULTS >>>
git1 (git) gitcheck
[no local mods, no repository changes]
git2 (git) gitcheck
onsub --chdir /private/tmp/all/wc/git2 --depth 1 --comment "wc=1,sh=0,out=1,in=1" {download-get}
<<< ERRORS >>>
(6) git2 (git) gitcheck
onsub --chdir /private/tmp/all/wc/git2 --depth 1 --comment "wc=1,sh=0,out=1,in=1" {download-get}
\end{lstlisting}
\end{snugshade}

Error \lstinline{6} indicates that the \lstinline{git2} folder has commits to sync from the remote server. We can execute the \lstinline{download-get} command to pull those commits:

\begin{snugshade}
\begin{lstlisting}[language=bash]	
$ onsub --chdir /private/tmp/all/wc/git2 --depth 1 --comment "wc=1,sh=0,out=1,in=1" {download-get}
. (git) git fetch ; git stash ; git rebase origin/master ; git mergetool -y --tool=kdiff3 ; git rebase --continue ; git merge origin/master ; git stash pop ; git mergetool -y --tool=kdiff3 ; git reset --mixed ; git stash clear
<<< RESULTS >>>
. (git) git fetch ; git stash ; git rebase origin/master ; git mergetool -y --tool=kdiff3 ; git rebase --continue ; git merge origin/master ; git stash pop ; git mergetool -y --tool=kdiff3 ; git reset --mixed ; git stash clear
Saved working directory and index state WIP on master: ff1f28a git2
First, rewinding head to replay your work on top of it...
Applying: git2
Using index info to reconstruct a base tree...
M       file.txt
Falling back to patching base and 3-way merge...
Auto-merging file.txt
CONFLICT (content): Merge conflict in file.txt
error: Failed to merge in the changes.
hint: Use 'git am --show-current-patch' to see the failed patch
Patch failed at 0001 git2
Resolve all conflicts manually, mark them as resolved with "git add/rm <conflicted_files>", then run "git rebase --continue".
You can instead skip this commit: run "git rebase --skip".
To abort and get back to the state before "git rebase", run "git rebase --abort".
Merging:
file.txt

Normal merge conflict for 'file.txt':
	{local}: modified file
	{remote}: modified file
Applying: git2
Already up to date.
Auto-merging file.txt
CONFLICT (content): Merge conflict in file.txt
The stash entry is kept in case you need it again.
Merging:
file.txt

Normal merge conflict for 'file.txt':
	{local}: modified file
	{remote}: modified file
Unstaged changes after reset:
M       file.txt

$ onsub --disable hg --disable svn check
git2 (git) gitcheck
git1 (git) gitcheck
<<< RESULTS >>>
git2 (git) gitcheck
onsub --chdir /private/tmp/all/wc/git2 --depth 1 --comment "wc=1,sh=0,out=1,in=0" {put-upload}
git1 (git) gitcheck
[no local mods, no repository changes]
<<< ERRORS >>>
(4) git2 (git) gitcheck
onsub --chdir /private/tmp/all/wc/git2 --depth 1 --comment "wc=1,sh=0,out=1,in=0" {put-upload}
\end{lstlisting}
\end{snugshade}

Error \lstinline{4} indicates that the \lstinline{git2} folder has local working copy changes and commits to push to the remote server. We can execute the \lstinline{put-upload} command to commit and push those changesets:

\begin{snugshade}
\begin{lstlisting}[language=bash]	
$ onsub --chdir /private/tmp/all/wc/git2 --depth 1 --comment "wc=1,sh=0,out=1,in=0" {put-upload}
. (git) git commit -a -e -m "# $(pwd)" ; git push
<<< RESULTS >>>
. (git) git commit -a -e -m "# $(pwd)" ; git push
[master 6c6bff9] git2
	1 file changed, 1 insertion(+), 1 deletion(-)
To /tmp/all/wc/../repo/git
	6c1e32b..6c6bff9  master -> master

$ onsub --disable hg --disable svn check
git2 (git) gitcheck
git1 (git) gitcheck
<<< RESULTS >>>
git2 (git) gitcheck
[no local mods, no repository changes]
git1 (git) gitcheck
onsub --chdir /private/tmp/all/wc/git1 --depth 1 --comment "wc=0,sh=0,out=0,in=2" {get}
<<< ERRORS >>>
(3) git1 (git) gitcheck
onsub --chdir /private/tmp/all/wc/git1 --depth 1 --comment "wc=0,sh=0,out=0,in=2" {get}
\end{lstlisting}
\end{snugshade}

Error \lstinline{3} indicates that the \lstinline{git1} folder has changesets to sync with the remote server. We're going to muck things up a bit, though, and create an outgoing changeset first:

\begin{snugshade}
\begin{lstlisting}[language=bash]	
$ cd git1
$ echo "git1 git1 git1" > file.txt
$ git ci -a -m "more git1"
[master 8027957] more git1
	1 file changed, 1 insertion(+), 1 deletion(-)
$ cd ..
$ onsub --disable hg --disable svn check
git2 (git) gitcheck
git1 (git) gitcheck
<<< RESULTS >>>
git2 (git) gitcheck
[no local mods, no repository changes]
git1 (git) gitcheck
onsub --chdir /private/tmp/all/wc/git1 --depth 1 --comment "wc=0,sh=0,out=1,in=2" {download}
<<< ERRORS >>>
(7) git1 (git) gitcheck
onsub --chdir /private/tmp/all/wc/git1 --depth 1 --comment "wc=0,sh=0,out=1,in=2" {download}
\end{lstlisting}
\end{snugshade}

Error \lstinline{7} indicates that the \lstinline{git1} folder has changesets to rebase with the remote server. We can execute the \lstinline{download} command to pull and rebase those changesets:

\begin{snugshade}
\begin{lstlisting}[language=bash]	
$ onsub --chdir /private/tmp/all/wc/git1 --depth 1 --comment "wc=0,sh=0,out=1,in=2" {download}
. (git) git fetch ; git stash ; git rebase origin/master ; git mergetool -y --tool=kdiff3 ; git rebase --continue
<<< RESULTS >>>
. (git) git fetch ; git stash ; git rebase origin/master ; git mergetool -y --tool=kdiff3 ; git rebase --continue
No local changes to save
First, rewinding head to replay your work on top of it...
Applying: more git1
Using index info to reconstruct a base tree...
M       file.txt
Falling back to patching base and 3-way merge...
Auto-merging file.txt
CONFLICT (content): Merge conflict in file.txt
error: Failed to merge in the changes.
hint: Use 'git am --show-current-patch' to see the failed patch
Patch failed at 0001 more git1
Resolve all conflicts manually, mark them as resolved with "git add/rm <conflicted_files>", then run "git rebase --continue".
You can instead skip this commit: run "git rebase --skip".
To abort and get back to the state before "git rebase", run "git rebase --abort".
Merging:
file.txt

Normal merge conflict for 'file.txt':
	{local}: modified file
	{remote}: modified file
Applying: more git1

$ onsub --disable hg --disable svn check
git2 (git) gitcheck
git1 (git) gitcheck
<<< RESULTS >>>
git2 (git) gitcheck
[no local mods, no repository changes]
git1 (git) gitcheck
onsub --chdir /private/tmp/all/wc/git1 --depth 1 --comment "wc=0,sh=0,out=1,in=0" {upload}
<<< ERRORS >>>
(5) git1 (git) gitcheck
onsub --chdir /private/tmp/all/wc/git1 --depth 1 --comment "wc=0,sh=0,out=1,in=0" {upload}
\end{lstlisting}
\end{snugshade}

Error \lstinline{5} again indicates that the \lstinline{git1} folder has commits to upload to the remote server. We can execute the \lstinline{upload} command to push those commits:

\begin{snugshade}
\begin{lstlisting}[language=bash]	
$ onsub --chdir /private/tmp/all/wc/git1 --depth 1 --comment "wc=0,sh=0,out=1,in=0" {upload}
. (git) git push
<<< RESULTS >>>
. (git) git push
To /tmp/all/wc/../repo/git
	6c6bff9..624c52a  master -> master

$ onsub --disable hg --disable svn check
git2 (git) gitcheck
git1 (git) gitcheck
<<< RESULTS >>>
git2 (git) gitcheck
onsub --chdir /private/tmp/all/wc/git2 --depth 1 --comment "wc=0,sh=0,out=0,in=1" {get}
git1 (git) gitcheck
[no local mods, no repository changes]
<<< ERRORS >>>
(3) git2 (git) gitcheck
onsub --chdir /private/tmp/all/wc/git2 --depth 1 --comment "wc=0,sh=0,out=0,in=1" {get}
\end{lstlisting}
\end{snugshade}

Error \lstinline{3} indicates that the \lstinline{git2} folder has commits to pull from the remote server. We can execute the \lstinline{get} command to pull those commits:

\begin{snugshade}
\begin{lstlisting}[language=bash]	
$ onsub --chdir /private/tmp/all/wc/git2 --depth 1 --comment "wc=0,sh=0,out=0,in=1" {get}
. (git) git stash pop ; git mergetool -y --tool=kdiff3 ; git reset --mixed ; git stash clear ; git stash ; git merge origin/master ; git stash pop ; git mergetool -y --tool=kdiff3 ; git reset --mixed ; git stash clear
<<< RESULTS >>>
. (git) git stash pop ; git mergetool -y --tool=kdiff3 ; git reset --mixed ; git stash clear ; git stash ; git merge origin/master ; git stash pop ; git mergetool -y --tool=kdiff3 ; git reset --mixed ; git stash clear
No stash entries found.
No files need merging
No local changes to save
Updating 6c6bff9..624c52a
Fast-forward
	file.txt | 2 +-
	1 file changed, 1 insertion(+), 1 deletion(-)
No stash entries found.
No files need merging

$ onsub --disable hg --disable svn check
git2 (git) gitcheck
git1 (git) gitcheck
<<< RESULTS >>>
git2 (git) gitcheck
[no local mods, no repository changes]
git1 (git) gitcheck
[no local mods, no repository changes]
\end{lstlisting}
\end{snugshade}

Now, all \Git\ repositories are up to date.

\subsection{\Mercurial}

\Mercurial\ can check the remote server without pulling but it does not track the remote branches after pulling. With those limitations and benefits, \lstinline{onsub} can only easily support four (4) operations: \lstinline{put}, \lstinline{upload}, \lstinline{put-upload} and \lstinline{download-get}:

\begin{itemize}
\item \lstinline{put} ~~~~~~~~ -- commits local changes (remote is unchanged)
\item \lstinline{upload} ~~~~~ -- pushes local commits (working copy and local are unchanged)
\item \lstinline{put-upload} ~ -- commits local changes and pushes local commits to remote
\item \lstinline{download-get} -- pulls, rebases and merges with working copy (remote is unchanged)
\end{itemize}

Determine the current state of the \hg\ clones with \lstinline|{check}|:

\begin{snugshade}
\begin{lstlisting}[language=bash]	
$ onsub --disable git --disable svn check
hg1 (hg) hgcheck
hg2 (hg) hgcheck
<<< RESULTS >>>
hg1 (hg) hgcheck
onsub --chdir /private/tmp/all/wc/hg1 --depth 1 --comment "wc=1,sh=0,out=0,in=0" {put}
hg2 (hg) hgcheck
onsub --chdir /private/tmp/all/wc/hg2 --depth 1 --comment "wc=1,sh=0,out=1,in=0" {put-upload}
<<< ERRORS >>>
(8) hg1 (hg) hgcheck
onsub --chdir /private/tmp/all/wc/hg1 --depth 1 --comment "wc=1,sh=0,out=0,in=0" {put}
(4) hg2 (hg) hgcheck
onsub --chdir /private/tmp/all/wc/hg2 --depth 1 --comment "wc=1,sh=0,out=1,in=0" {put-upload}
\end{lstlisting}
\end{snugshade}

Error \lstinline{8} indicates that the \lstinline{hg1} folder has local changes to commit. The output command for that error instructs how those folders might be brought to a clean working copy and synced local clone.

We can execute the \lstinline{put} command to commit the working copy changes:

\begin{snugshade}
\begin{lstlisting}[language=bash]	
$ onsub --chdir /private/tmp/all/wc/hg1 --depth 1 --comment "wc=1,sh=0,out=0,in=0" {put}
. (hg) hg commit -e -m "HG: $(pwd)"
<<< RESULTS >>>
. (hg) hg commit -e -m "HG: $(pwd)"

$ onsub --disable git --disable svn check
hg1 (hg) hgcheck
hg2 (hg) hgcheck
<<< RESULTS >>>
hg2 (hg) hgcheck
onsub --chdir /private/tmp/all/wc/hg2 --depth 1 --comment "wc=1,sh=0,out=1,in=0" {put-upload}
hg1 (hg) hgcheck
onsub --chdir /private/tmp/all/wc/hg1 --depth 1 --comment "wc=0,sh=0,out=1,in=0" {upload}
<<< ERRORS >>>
(4) hg2 (hg) hgcheck
onsub --chdir /private/tmp/all/wc/hg2 --depth 1 --comment "wc=1,sh=0,out=1,in=0" {put-upload}
(5) hg1 (hg) hgcheck
onsub --chdir /private/tmp/all/wc/hg1 --depth 1 --comment "wc=0,sh=0,out=1,in=0" {upload}
\end{lstlisting}
\end{snugshade}

Error \lstinline{5} indicates that the \lstinline{hg1} folder has commits to upload to the remote server. We can execute the \lstinline{upload} command to push those commits:

\begin{snugshade}
\begin{lstlisting}[language=bash]	
$ onsub --chdir /private/tmp/all/wc/hg1 --depth 1 --comment "wc=0,sh=0,out=1,in=0" {upload}
. (hg) hg push
<<< RESULTS >>>
. (hg) hg push
pushing to /private/tmp/all/repo/hg
searching for changes
adding changesets
adding manifests
adding file changes
added 1 changesets with 1 changes to 1 files

$ onsub --disable git --disable svn check
hg1 (hg) hgcheck
hg2 (hg) hgcheck
<<< RESULTS >>>
hg1 (hg) hgcheck
[no local mods, no repository changes]
hg2 (hg) hgcheck
onsub --chdir /private/tmp/all/wc/hg2 --depth 1 --comment "wc=1,sh=0,out=1,in=1" {download-get}
<<< ERRORS >>>
(6) hg2 (hg) hgcheck
onsub --chdir /private/tmp/all/wc/hg2 --depth 1 --comment "wc=1,sh=0,out=1,in=1" {download-get}
\end{lstlisting}
\end{snugshade}

Error \lstinline{6} indicates that the \lstinline{hg2} folder has commits to sync from the remote server. We can execute the \lstinline{download-get} command to pull those commits:

\begin{snugshade}
\begin{lstlisting}[language=bash]	
$ onsub --chdir /private/tmp/all/wc/hg2 --depth 1 --comment "wc=1,sh=0,out=1,in=1" {download-get}
. (hg) hg shelve ; hg pull --rebase ; hg update ; hg unshelve
<<< RESULTS >>>
. (hg) hg shelve ; hg pull --rebase ; hg update ; hg unshelve
shelved as default
1 files updated, 0 files merged, 0 files removed, 0 files unresolved
pulling from /private/tmp/all/repo/hg
searching for changes
adding changesets
adding manifests
adding file changes
added 1 changesets with 1 changes to 1 files (+1 heads)
new changesets 0adefd4c3fbb
rebasing 1:f3fd648e6321 "hg2"
merging file.txt
running merge tool kdiff3 for file file.txt
saved backup bundle to /private/tmp/all/wc/hg2/.hg/strip-backup/f3fd648e6321-a76485e0-rebase.hg
0 files updated, 0 files merged, 0 files removed, 0 files unresolved
unshelving change 'default'
rebasing shelved changes
merging file.txt
running merge tool kdiff3 for file file.txt

$ onsub --disable git --disable svn check
hg1 (hg) hgcheck
hg2 (hg) hgcheck
<<< RESULTS >>>
hg2 (hg) hgcheck
onsub --chdir /private/tmp/all/wc/hg2 --depth 1 --comment "wc=1,sh=0,out=1,in=0" {put-upload}
hg1 (hg) hgcheck
[no local mods, no repository changes]
<<< ERRORS >>>
(4) hg2 (hg) hgcheck
onsub --chdir /private/tmp/all/wc/hg2 --depth 1 --comment "wc=1,sh=0,out=1,in=0" {put-upload}
\end{lstlisting}
\end{snugshade}

Error \lstinline{4} indicates that the \lstinline{hg2} folder has local working copy changes and commits to push to the remote server. We can execute the \lstinline{put-upload} command to commit and push those changesets:

\begin{snugshade}
\begin{lstlisting}[language=bash]	
$ onsub --chdir /private/tmp/all/wc/hg2 --depth 1 --comment "wc=1,sh=0,out=1,in=0" {put-upload}
. (hg) hg commit -e -m "HG: $(pwd)" ; hg push
<<< RESULTS >>>
. (hg) hg commit -e -m "HG: $(pwd)" ; hg push
pushing to /private/tmp/all/repo/hg
searching for changes
adding changesets
adding manifests
adding file changes
added 2 changesets with 2 changes to 1 files

$ onsub --disable git --disable svn check
hg1 (hg) hgcheck
hg2 (hg) hgcheck
<<< RESULTS >>>
hg2 (hg) hgcheck
[no local mods, no repository changes]
hg1 (hg) hgcheck
onsub --chdir /private/tmp/all/wc/hg1 --depth 1 --comment "wc=0,sh=0,out=0,in=2" {download-get}
<<< ERRORS >>>
(3) hg1 (hg) hgcheck
onsub --chdir /private/tmp/all/wc/hg1 --depth 1 --comment "wc=0,sh=0,out=0,in=2" {download-get}
\end{lstlisting}
\end{snugshade}

Error \lstinline{3} again indicates that the \lstinline{hg1} folder has commits to sync from the remote server. Executing one last \lstinline{download-get} command:

\begin{snugshade}
\begin{lstlisting}[language=bash]	
$ onsub --chdir /private/tmp/all/wc/hg1 --depth 1 --comment "wc=0,sh=0,out=0,in=2" {download-get}
. (hg) hg shelve ; hg pull --rebase ; hg update ; hg unshelve
<<< RESULTS >>>
. (hg) hg shelve ; hg pull --rebase ; hg update ; hg unshelve
nothing changed
pulling from /private/tmp/all/repo/hg
searching for changes
adding changesets
adding manifests
adding file changes
added 2 changesets with 2 changes to 1 files
new changesets 0151632719df:24003a29bcf6
nothing to rebase - updating instead
1 files updated, 0 files merged, 0 files removed, 0 files unresolved
0 files updated, 0 files merged, 0 files removed, 0 files unresolved
abort: no shelved changes to apply!
<<< ERRORS >>>
(255) . (hg) hg shelve ; hg pull --rebase ; hg update ; hg unshelve
nothing changed
pulling from /private/tmp/all/repo/hg
searching for changes
adding changesets
adding manifests
adding file changes
added 2 changesets with 2 changes to 1 files
new changesets 0151632719df:24003a29bcf6
nothing to rebase - updating instead
1 files updated, 0 files merged, 0 files removed, 0 files unresolved
0 files updated, 0 files merged, 0 files removed, 0 files unresolved
abort: no shelved changes to apply!

$ onsub --disable git --disable svn check
hg1 (hg) hgcheck
hg2 (hg) hgcheck
<<< RESULTS >>>
hg1 (hg) hgcheck
[no local mods, no repository changes]
hg2 (hg) hgcheck
[no local mods, no repository changes]
\end{lstlisting}
\end{snugshade}

Now, all \Mercurial\ repositories are up to date.

\subsection{\Subversion}

\Subversion\ does not support local changesets. With that limitation, \lstinline{onsub} can only support two (2) operations: \lstinline{put-upload} and \lstinline{download-get}:

\begin{itemize}
\item \lstinline{put-upload} ~ -- commits local to remote
\item \lstinline{download-get} -- merges with working copy (remote is unchanged)
\end{itemize}

Determine the current state of the \svn\ clones with \lstinline|{check}|:

\begin{snugshade}
\begin{lstlisting}[language=bash]	
$ onsub --disable git --disable hg check
svn2 (svn) svncheck
svn1 (svn) svncheck
<<< RESULTS >>>
svn2 (svn) svncheck
onsub --chdir /private/tmp/all/wc/svn2 --depth 1 --comment "wc=1,in=0" {put-upload}
svn1 (svn) svncheck
onsub --chdir /private/tmp/all/wc/svn1 --depth 1 --comment "wc=1,in=0" {put-upload}
<<< ERRORS >>>
(3) svn2 (svn) svncheck
onsub --chdir /private/tmp/all/wc/svn2 --depth 1 --comment "wc=1,in=0" {put-upload}
(3) svn1 (svn) svncheck
onsub --chdir /private/tmp/all/wc/svn1 --depth 1 --comment "wc=1,in=0" {put-upload}
\end{lstlisting}
\end{snugshade}

Error \lstinline{3} indicates that the \lstinline{svn1} and \lstinline{svn2} folders have local changes to commit. The output command for that error instructs how those folders might be brought to a clean working copy and synced local clone.

We can execute the \lstinline{put-upload} command to sync the \lstinline{svn1} working copy with the remote server:

\begin{snugshade}
\begin{lstlisting}[language=bash]	
$ onsub --chdir /private/tmp/all/wc/svn1 --depth 1 --comment "wc=1,in=0" {put-upload}
. (svn) svn commit
<<< RESULTS >>>
. (svn) svn commit
Sending        file.txt
Transmitting file data .done
Committing transaction...
Committed revision 2.

$ onsub --disable git --disable hg check
svn2 (svn) svncheck
svn1 (svn) svncheck
<<< RESULTS >>>
svn1 (svn) svncheck
[no local mods, no repository changes]
svn2 (svn) svncheck
onsub --chdir /private/tmp/all/wc/svn2 --depth 1 --comment "wc=1,in=1" {download-get}
<<< ERRORS >>>
(2) svn2 (svn) svncheck
onsub --chdir /private/tmp/all/wc/svn2 --depth 1 --comment "wc=1,in=1" {download-get}
\end{lstlisting}
\end{snugshade}

Error \lstinline{2} indicates that the \lstinline{svn2} working copy has remote changes to merge. We can execute a \lstinline{download-get} command to merge with the \lstinline{svn2} working copy:

\begin{snugshade}
\begin{lstlisting}[language=bash]	
$ onsub --chdir /private/tmp/all/wc/svn2 --depth 1 --comment "wc=1,in=1" {download-get}
. (svn) svn update --accept l
<<< RESULTS >>>
. (svn) svn update --accept l
Updating '.':
C    file.txt
Updated to revision 2.
Merge conflicts in 'file.txt' marked as resolved.
Summary of conflicts:
	Text conflicts: 0 remaining (and 1 already resolved)

$ onsub --disable git --disable hg check
svn2 (svn) svncheck
svn1 (svn) svncheck
<<< RESULTS >>>
svn2 (svn) svncheck
onsub --chdir /private/tmp/all/wc/svn2 --depth 1 --comment "wc=1,in=0" {put-upload}
svn1 (svn) svncheck
[no local mods, no repository changes]
<<< ERRORS >>>
(3) svn2 (svn) svncheck
onsub --chdir /private/tmp/all/wc/svn2 --depth 1 --comment "wc=1,in=0" {put-upload}
\end{lstlisting}
\end{snugshade}

Error \lstinline{3} again indicates that the \lstinline{svn2} folder has local changes to sync with the remote server. We can execute another \lstinline{put-upload} command to sync the \lstinline{svn2} working copy with the remote server:

\begin{snugshade}
\begin{lstlisting}[language=bash]	
$ onsub --chdir /private/tmp/all/wc/svn2 --depth 1 --comment "wc=1,in=0" {put-upload}
. (svn) svn commit
<<< RESULTS >>>
. (svn) svn commit
Sending        file.txt
Transmitting file data .done
Committing transaction...
Committed revision 3.

$ onsub --disable git --disable hg check
svn2 (svn) svncheck
svn1 (svn) svncheck
<<< RESULTS >>>
svn2 (svn) svncheck
[no local mods, no repository changes]
svn1 (svn) svncheck
onsub --chdir /private/tmp/all/wc/svn1 --depth 1 --comment "wc=0,in=1" {download-get}
<<< ERRORS >>>
(2) svn1 (svn) svncheck
onsub --chdir /private/tmp/all/wc/svn1 --depth 1 --comment "wc=0,in=1" {download-get}
\end{lstlisting}
\end{snugshade}

Error \lstinline{2} again indicates that the \lstinline{svn1} folder has remote changes to update. We can execute another \lstinline{download-get} command to sync the \lstinline{svn1} working copy with the remote server:

\begin{snugshade}
\begin{lstlisting}[language=bash]	
$ onsub --chdir /private/tmp/all/wc/svn1 --depth 1 --comment "wc=0,in=1" {download-get}
. (svn) svn update --accept l
<<< RESULTS >>>
. (svn) svn update --accept l
Updating '.':
U    file.txt
Updated to revision 3.

$ onsub --disable git --disable hg check
svn2 (svn) svncheck
svn1 (svn) svncheck
<<< RESULTS >>>
svn2 (svn) svncheck
[no local mods, no repository changes]
svn1 (svn) svncheck
[no local mods, no repository changes]
\end{lstlisting}
\end{snugshade}

Now, all \Subversion\ working copies are up to date.

\section{Notes}

The code is completely undocumented right now but it's pretty short and leverages a bunch of \Python\ magic to provide a ton of flexibility. It can be compiled into executables as well but we can provide those later if this approach gains traction.

\end{document}
